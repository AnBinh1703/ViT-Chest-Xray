% ============================================================
% CHAPTER 3: DATASET - NIH CHEST X-RAY ANALYSIS
% ============================================================
\chapter{Dataset Analysis: NIH Chest X-ray}

\section{Dataset Overview}

\paperref{Section 4.1 - Dataset}

\subsection{Basic Information}

\begin{tcolorbox}[colback=blue!5!white,colframe=blue!75!black,title=NIH Chest X-ray Dataset]
\begin{itemize}
    \item \textbf{Full name:} ChestX-ray14 (NIH Clinical Center)
    \item \textbf{Number of images:} 112,120 frontal-view X-ray images
    \item \textbf{Number of patients:} 30,805 unique patients
    \item \textbf{Number of labels:} 14 diseases + 1 ``No Finding'' = 15 classes
    \item \textbf{Label source:} Text-mining from radiology reports (weak labels)
    \item \textbf{Original image size:} 1024 × 1024 pixels
\end{itemize}
\end{tcolorbox}

\subsection{Paper Citation}

\begin{tcolorbox}[colback=yellow!5!white,colframe=yellow!75!black,title=Paper Quote - Dataset]
\textit{``To evaluate the performance of our model architectures, we utilized two freely available datasets: the NIH Chest X-ray dataset comprising 112,120 X-ray images with disease labels from 30,805 unique patients, and a Random Sample of the NIH Chest X-ray Dataset, containing 5,606 X-ray images. Both datasets involved multi-class classification across 15 classes, each representing different disease labels.''}
\end{tcolorbox}

\section{The 15 Classes}

\subsection{Disease Classification}

\begin{table}[H]
\centering
\caption{15 Classes in NIH Chest X-ray Dataset}
\begin{tabular}{clcp{6cm}}
\toprule
\textbf{ID} & \textbf{Disease Name} & \textbf{Prevalence (\%)} & \textbf{Description} \\
\midrule
0 & Cardiomegaly & 2.48 & Enlarged heart \\
1 & Emphysema & 2.24 & Lung tissue destruction \\
2 & Effusion & 11.88 & Pleural effusion \\
3 & Hernia & 0.20 & Hiatal hernia \\
4 & Nodule & 5.65 & Lung nodule \\
5 & Pneumothorax & 4.73 & Collapsed lung \\
6 & Atelectasis & 10.31 & Lung collapse \\
7 & Pleural\_Thickening & 3.02 & Pleural thickening \\
8 & Mass & 5.16 & Lung mass \\
9 & Edema & 2.05 & Pulmonary edema \\
10 & Consolidation & 4.16 & Lung consolidation \\
11 & Infiltration & 17.74 & Lung infiltration \\
12 & Fibrosis & 1.50 & Pulmonary fibrosis \\
13 & Pneumonia & 1.28 & Pneumonia \\
14 & No Finding & 53.84 & No disease detected \\
\bottomrule
\end{tabular}
\end{table}

\subsection{Class Imbalance Analysis}

\begin{tcolorbox}[colback=red!5!white,colframe=red!75!black,title=Class Imbalance Problem]
\textbf{Key observations:}
\begin{itemize}
    \item ``No Finding'' accounts for \textbf{53.84\%} - more than half the dataset
    \item ``Hernia'' accounts for only \textbf{0.20\%} - extremely rare
    \item Highest/lowest ratio = 53.84 / 0.20 = \textbf{269 times}
\end{itemize}

\textbf{Consequences:}
\begin{itemize}
    \item Model may be biased toward predicting ``No Finding''
    \item High accuracy does not necessarily mean good detection of rare diseases
    \item Metrics like AUC are needed instead of just accuracy
\end{itemize}
\end{tcolorbox}

\section{Multi-label Nature}

\subsection{Multi-label vs Multi-class}

\begin{table}[H]
\centering
\caption{Comparison: Multi-class vs Multi-label Classification}
\begin{tabular}{lp{5cm}p{5cm}}
\toprule
\textbf{Aspect} & \textbf{Multi-class} & \textbf{Multi-label} \\
\midrule
Labels per sample & Exactly 1 & Can be multiple (0, 1, 2, ...) \\
Output activation & Softmax & Sigmoid (independent) \\
Loss function & Categorical CE & Binary CE \\
Example & Cat OR Dog OR Bird & Cat AND Dog (both possible) \\
NIH Dataset & Not suitable & \checkmark Suitable \\
\bottomrule
\end{tabular}
\end{table}

\subsection{Multi-label Example in NIH}

An X-ray image can have multiple diseases simultaneously:

\begin{lstlisting}[caption={Multi-label example in dataset}]
# Image 00000001_000.png may have labels:
labels = [0, 0, 1, 0, 0, 0, 1, 0, 0, 0, 0, 0, 0, 0, 0]
#         ^     ^        ^
#         |     |        |
#         |     |        Atelectasis (index 6)
#         |     Effusion (index 2)
#         Cardiomegaly (index 0) - Not present

# Patient has: Effusion + Atelectasis (2 diseases simultaneously)
\end{lstlisting}

\section{Data Pipeline in Code}

\coderef{data.ipynb}

\subsection{Loading and Preprocessing}

\begin{lstlisting}[caption={Data loading from data.ipynb (PyTorch version)}]
class ChestXrayDataset(Dataset):
    def __init__(self, dataframe, images_path, labels, transform=None):
        self.dataframe = dataframe.reset_index(drop=True)
        self.images_path = images_path
        self.labels = labels
        self.transform = transform
        
    def __len__(self):
        return len(self.dataframe)
    
    def __getitem__(self, idx):
        # Load image
        img_name = self.dataframe.iloc[idx]['Image Index']
        img_path = os.path.join(self.images_path, img_name)
        image = Image.open(img_path).convert('RGB')
        
        # Apply transforms
        if self.transform:
            image = self.transform(image)
        
        # Get labels as one-hot vector
        label = torch.tensor(
            self.dataframe.iloc[idx][self.labels].values.astype(float),
            dtype=torch.float32
        )
        
        return image, label
\end{lstlisting}

\subsection{Data Augmentation}

\paperref{Section 4.2 - Models}

\begin{tcolorbox}[colback=green!5!white,colframe=green!75!black,title=Augmentation from Paper]
\textit{``We also performed various data augmentations on both datasets. For the Chest X-ray dataset, we applied resizing, random horizontal flip, and random rotation.''}
\end{tcolorbox}

\begin{lstlisting}[caption={Data augmentation transforms}]
train_transform = transforms.Compose([
    transforms.Resize((224, 224)),        # Resize to standard size
    transforms.RandomHorizontalFlip(p=0.5), # Random flip
    transforms.RandomRotation(degrees=5),   # Small rotation
    transforms.ColorJitter(brightness=0.1, contrast=0.1),
    transforms.ToTensor(),
    transforms.Normalize(
        mean=[0.485, 0.456, 0.406],  # ImageNet stats
        std=[0.229, 0.224, 0.225]
    )
])

val_transform = transforms.Compose([
    transforms.Resize((224, 224)),
    transforms.ToTensor(),
    transforms.Normalize(
        mean=[0.485, 0.456, 0.406],
        std=[0.229, 0.224, 0.225]
    )
])
\end{lstlisting}

\subsection{Note on Horizontal Flip in X-ray}

\begin{tcolorbox}[colback=orange!5!white,colframe=orange!75!black,title=Warning: Horizontal Flip]
In X-ray images, horizontal flip can cause issues:
\begin{itemize}
    \item Heart is normally on the left → flip makes heart appear on right (Dextrocardia - abnormal)
    \item Some diseases have ``laterality'' (different left/right characteristics)
\end{itemize}

\textbf{Recommendation:} Ablation study needed to evaluate flip impact.
\end{tcolorbox}

\section{Data Split}

\subsection{Paper Description}

\paperref{Section 4.1 - Dataset}

\begin{tcolorbox}[colback=yellow!5!white,colframe=yellow!75!black,title=Data Split from Paper]
\textit{``However, our model training was conducted on a subset of 85,000 images from this Random Sample Dataset. We observed a faster convergence to optimal outputs within this subset.''}
\end{tcolorbox}

\subsection{Implementation in Code}

\begin{lstlisting}[caption={Data split implementation}]
from sklearn.model_selection import train_test_split

# Standard split: 60% train, 20% val, 20% test
train_df, temp_df = train_test_split(
    full_df, 
    test_size=0.4, 
    random_state=42
)
val_df, test_df = train_test_split(
    temp_df, 
    test_size=0.5, 
    random_state=42
)

print(f"Training samples: {len(train_df)}")
print(f"Validation samples: {len(val_df)}")
print(f"Test samples: {len(test_df)}")
\end{lstlisting}

\subsection{Data Leakage Concern}

\begin{tcolorbox}[colback=red!5!white,colframe=red!75!black,title=Warning: Patient-level Split]
\textbf{Issue:} Paper does not clearly mention patient-level split.

\textbf{Risk:} If split by image (not by patient):
\begin{itemize}
    \item Same patient may have multiple images
    \item Images from same patient may appear in both train and test
    \item Model may ``memorize'' patients instead of learning disease features
    \item Evaluation results may be inflated (artificially high)
\end{itemize}

\textbf{Recommendation:} Split by Patient ID to ensure true generalization.
\end{tcolorbox}

\section{Dataset Statistics Used in Experiments}

\begin{table}[H]
\centering
\caption{Dataset splits in experiments}
\begin{tabular}{lccc}
\toprule
\textbf{Experiment} & \textbf{Train} & \textbf{Val} & \textbf{Test} \\
\midrule
Paper (Full) & 68,000 & 8,500 & 8,500 \\
Paper (Subset) & 3,363 & 1,121 & 1,122 \\
Our experiment (demo) & 60 & 20 & 20 \\
\bottomrule
\end{tabular}
\end{table}

\textbf{Note:} Our demo experiment uses a very small dataset (100 samples) for pipeline testing. Results are not reliable for model evaluation.
