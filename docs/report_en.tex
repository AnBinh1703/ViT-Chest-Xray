% ============================================================
%  COMPREHENSIVE ANALYSIS REPORT — ENGLISH VERSION
%  Paper: A Comparative Study of CNN, ResNet, and Vision Transformers
%         for Multi-Classification of Chest Diseases
%  Compile: pdfLaTeX, UTF-8
% ============================================================

\documentclass[12pt,a4paper]{report}

% === Encoding & Language ===
\usepackage[utf8]{inputenc}
\usepackage[T1]{fontenc}
\usepackage[english]{babel}

% === Fonts ===
\usepackage{newtxtext}

% === Page layout ===
\usepackage[a4paper,top=2cm,bottom=2cm,left=2.5cm,right=2cm]{geometry}
\usepackage{setspace}\onehalfspacing
\usepackage{parskip}
\usepackage{indentfirst}
\usepackage{microtype}

% === Section formatting ===
\usepackage{titlesec}
\titleformat{\chapter}{\normalfont\huge\bfseries\color{blue!70!black}}{\thechapter.}{0.5em}{\Huge\color{blue!70!black}}
\titlespacing*{\chapter}{0pt}{-10pt}{10pt}
\titleformat{\section}{\normalfont\Large\bfseries\color{blue!60!black}}{\thesection}{1em}{}
\titleformat{\subsection}{\normalfont\large\bfseries\color{blue!50!black}}{\thesubsection}{1em}{}

% === Math / Tables / Figures ===
\usepackage{amsmath,amssymb}
\usepackage{booktabs,longtable,array,multirow}
\usepackage{graphicx,float,subcaption}
\usepackage{caption}
\captionsetup{font={small,it},labelfont={bf,color=blue!70!black}}
\usepackage{algorithm,algorithmic}
\DeclareGraphicsExtensions{.pdf,.png,.jpg,.jpeg}

% === TikZ ===
\usepackage{tikz}
\usetikzlibrary{shapes,arrows,positioning,calc}

% === Links ===
\usepackage{xcolor}
\usepackage{hyperref}
\hypersetup{colorlinks=true,linkcolor=blue!50!black,citecolor=green!50!black,urlcolor=blue}

% === Headers ===
\usepackage{fancyhdr}
\pagestyle{fancy}
\fancyhf{}
\fancyhead[L]{\small\leftmark}
\fancyhead[R]{\small\thepage}
\renewcommand{\headrulewidth}{0.4pt}

% === Lists / Code ===
\usepackage{enumitem}
\usepackage{listings}
\usepackage{tcolorbox}
\tcbuselibrary{listings,skins,breakable}

\definecolor{codegreen}{rgb}{0,0.6,0}
\definecolor{codegray}{rgb}{0.5,0.5,0.5}
\definecolor{codepurple}{rgb}{0.58,0,0.82}
\definecolor{backcolour}{rgb}{0.95,0.95,0.92}
\lstdefinestyle{pythonstyle}{
  backgroundcolor=\color{backcolour},
  commentstyle=\color{codegreen},
  keywordstyle=\color{blue},
  stringstyle=\color{codepurple},
  basicstyle=\ttfamily\small,
  breaklines=true,
  numbers=left,
  numberstyle=\tiny\color{codegray},
  frame=single,
  language=Python
}
\lstset{style=pythonstyle}

% === Macros ===
\newcommand{\paperref}[1]{\textcolor{blue!70!black}{[Paper: #1]}}

% ============================================================
\begin{document}

% === TITLE PAGE ===
\begin{titlepage}
\centering
\vspace*{1cm}
{\Large\textbf{FPT UNIVERSITY -- GRADUATE SCHOOL}}\\[0.3cm]
{\large Master of Software Engineering}\\[2cm]

{\Huge\bfseries\color{blue!70!black}
A Comparative Study of CNN, ResNet,\\and Vision Transformers\\for Multi-label Classification\\of Chest X-ray Diseases}\\[1.5cm]

{\Large Course: \textbf{DLE501 -- Deep Learning}}\\[0.5cm]
{\large Instructor: \textbf{Dr. [Instructor Name]}}\\[1cm]

\begin{tabular}{ll}
\textbf{Group 1:} & \\
& [Member 1] \\
& [Member 2] \\
& [Member 3] \\
\end{tabular}\\[2cm]

{\large\textbf{Ho Chi Minh City -- 2026}}
\end{titlepage}

% === ABSTRACT ===
\chapter*{Abstract}
\addcontentsline{toc}{chapter}{Abstract}

\begin{tcolorbox}[enhanced,colback=blue!3!white,colframe=blue!60!black,boxrule=1pt,arc=3mm,title={\textbf{\large Overview}}]
The paper \textit{``A Comparative Study of CNN, ResNet, and Vision Transformers for Multi-Classification of Chest Diseases''} (arXiv:2406.00237) compares the performance of three deep learning architectures for multi-label chest X-ray disease classification using the \textbf{NIH ChestX-ray14} dataset.

Models studied: CNN baseline, ResNet-34, ViT-v1 (from scratch), ViT-v2 (improved), and ViT-ResNet (pretrained).
\end{tcolorbox}

\vspace{0.5cm}

\begin{tcolorbox}[enhanced,colback=green!3!white,colframe=green!60!black,boxrule=1pt,arc=3mm,title={\textbf{\large Key Experimental Results}}]
\begin{table}[H]
\centering\small
\begin{tabular}{lccccc}
\toprule
\textbf{Model} & \textbf{Params} & \textbf{Test AUC} & \textbf{Test Acc} & \textbf{Epochs} & \textbf{Notes} \\
\midrule
CNN & $\sim$95.6M & 0.5777 & -- & 10 & Severe overfitting \\
ResNet-34 & $\sim$21.3M & 0.4462 & -- & 10 & Underfitting \\
ViT-v1 & 9.0M & 0.5854 & 91.33\% & 10 & Trained from scratch \\
ViT-v2 & 9.0M & 0.6303 & 89.67\% & 9 & SGD + Early stopping \\
ViT-ResNet & 85.8M & 0.6694 & 87.00\% & 10 & ImageNet pretrained \\
\midrule
\textbf{ViT Final} & \textbf{9.0M} & \textbf{0.7225} & \textbf{92.91\%} & \textbf{10} & \textbf{Full dataset 112K} \\
\bottomrule
\end{tabular}
\end{table}

\textit{On a small subset (60 training images), the pretrained ViT-ResNet achieved the highest AUC (0.6694). When trained on the full 112,120 images with patient-level splitting, ViT achieved AUC 0.7225.}
\end{tcolorbox}

% === TOC ===
\tableofcontents\clearpage
\listoffigures\clearpage
\listoftables\clearpage

\pagenumbering{arabic}\setcounter{page}{1}

% ============================================================
% CHAPTER 1: INTRODUCTION
% ============================================================
\chapter{Introduction}
\label{chap:intro}

\section{Clinical Background}
Pulmonary and cardiopulmonary diseases are among the leading causes of death globally. Chest X-ray imaging remains the most widely used diagnostic tool due to its low cost, speed, and ability to detect a wide range of pathologies.

\section{Challenges}
\begin{itemize}
  \item Shortage of radiologists, particularly in rural and developing regions.
  \item Inter-observer variability in image interpretation.
  \item Multi-label nature: a single image may contain multiple coexisting diseases.
  \item Subtle disease signs that are difficult to detect in early stages.
\end{itemize}

\section{Research Objectives}
This paper compares CNN, ResNet-34, and Vision Transformer on the NIH ChestX-ray14 dataset (112,120 images, 14 pathologies). Research questions:
\begin{enumerate}
  \item[\textbf{RQ1}] How do CNN, ResNet, and ViT compare in performance?
  \item[\textbf{RQ2}] How much does transfer learning improve over training from scratch?
  \item[\textbf{RQ3}] What matters more: architecture or data scale?
\end{enumerate}

% ============================================================
% CHAPTER 2: DATASET
% ============================================================
\chapter{NIH ChestX-ray14 Dataset}
\label{chap:dataset}

\section{Overview}

\begin{table}[H]
\centering
\caption{NIH ChestX-ray14 Dataset Summary}
\begin{tabular}{ll}
\toprule
\textbf{Attribute} & \textbf{Value} \\
\midrule
Total images & 112,120 \\
Unique patients & 30,805 \\
Number of classes & 14 + No Finding = 15 \\
Original resolution & 1024 $\times$ 1024 pixels \\
Format & PNG (grayscale $\to$ RGB) \\
\bottomrule
\end{tabular}
\end{table}

\section{Disease Classes}

\begin{table}[H]
\centering
\caption{15 Classes in NIH ChestX-ray14}
\label{tab:15classes}
\begin{tabular}{clcl}
\toprule
\textbf{ID} & \textbf{Disease} & \textbf{Prevalence (\%)} & \textbf{Description} \\
\midrule
0 & Cardiomegaly & 2.48 & Enlarged heart \\
1 & Emphysema & 2.24 & Air trapping \\
2 & Effusion & 11.88 & Pleural fluid \\
3 & Hernia & 0.20 & Diaphragmatic hernia \\
4 & Nodule & 5.65 & Lung nodule \\
5 & Pneumothorax & 4.73 & Collapsed lung \\
6 & Atelectasis & 10.31 & Lung collapse \\
7 & Pleural\_Thickening & 3.02 & Pleural thickening \\
8 & Mass & 5.16 & Lung mass \\
9 & Edema & 2.05 & Pulmonary edema \\
10 & Consolidation & 4.16 & Lung consolidation \\
11 & Infiltration & 17.74 & Lung infiltration \\
12 & Fibrosis & 1.50 & Pulmonary fibrosis \\
13 & Pneumonia & 1.28 & Pneumonia \\
14 & No Finding & 53.84 & Normal \\
\bottomrule
\end{tabular}
\end{table}

\section{Class Imbalance}
\begin{tcolorbox}[colback=red!5!white,colframe=red!75!black,title=Class Imbalance Problem]
\begin{itemize}
  \item ``No Finding'' accounts for \textbf{53.84\%} --- more than half the dataset.
  \item ``Hernia'' accounts for only \textbf{0.20\%} --- the rarest class.
  \item Maximum/minimum prevalence ratio = 53.84 / 0.20 = \textbf{269$\times$}.
\end{itemize}
\end{tcolorbox}

\section{Multi-label Classification}
Each image may carry multiple disease labels simultaneously. Therefore, \textbf{Sigmoid} activation (not Softmax) and \textbf{BCEWithLogitsLoss} are used.

\section{Data Splitting}

\textbf{Experiment 1 (Small-scale):} 60 train / 20 val / 20 test images.

\textbf{Experiment 2 (Full-scale):} 112,120 images split by Patient ID:
\begin{itemize}
  \item Train: 78,614 images (21,563 patients)
  \item Val: 11,212 images (3,081 patients)
  \item Test: 22,294 images (6,161 patients)
\end{itemize}

% ============================================================
% CHAPTER 3: CNN
% ============================================================
\chapter{Convolutional Neural Network (CNN)}
\label{chap:cnn}

\section{Architecture}
The CNN baseline consists of 2 convolutional layers (32 and 64 filters, $3\times3$ kernel), each followed by ReLU and $2\times2$ MaxPool, then Flatten, FC with 512 units, Dropout 0.5, and a 15-unit output layer.

\section{Parameter Analysis}
\begin{table}[H]
\centering
\caption{CNN Baseline Parameters}
\begin{tabular}{lcc}
\toprule
\textbf{Layer} & \textbf{Parameters} & \textbf{Ratio} \\
\midrule
Conv layers & $\sim$19,400 & 0.02\% \\
FC layers & $\sim$95,560,000 & 99.98\% \\
\midrule
\textbf{Total} & $\sim$\textbf{95.6M} & 100\% \\
\bottomrule
\end{tabular}
\end{table}

\begin{tcolorbox}[colback=red!5!white,colframe=red!75!black,title=Issue]
99\% of parameters reside in the FC layer $\Rightarrow$ severe overfitting. Train AUC = 0.9116 but Test AUC only reaches 0.5777.
\end{tcolorbox}

\section{Experimental Results}

\begin{table}[H]
\centering
\caption{CNN Results (small-scale, 60 training images)}
\begin{tabular}{lccc}
\toprule
\textbf{Metric} & \textbf{Train} & \textbf{Val} & \textbf{Test} \\
\midrule
Loss & 0.2034 & 0.2668 & -- \\
AUC & 0.9116 & 0.5777 & \textbf{0.5777} \\
\bottomrule
\end{tabular}
\end{table}

\begin{figure}[H]
\centering
\includegraphics[width=0.8\textwidth]{../assets/image/output_cnn.png}
\caption{CNN training history: clear overfitting}
\label{fig:cnn}
\end{figure}

% ============================================================
% CHAPTER 4: RESNET
% ============================================================
\chapter{Residual Network (ResNet-34)}
\label{chap:resnet}

\section{Core Idea: Skip Connections}
ResNet addresses the degradation problem through residual learning:
\[y = F(x) + x\]
The gradient always has a ``1'' component flowing through the skip connection, preventing vanishing gradients.

\section{ResNet-34 Architecture}
\begin{table}[H]
\centering
\caption{ResNet-34 Architecture}
\begin{tabular}{lcccc}
\toprule
\textbf{Stage} & \textbf{Output} & \textbf{Channels} & \textbf{Blocks} & \textbf{Stride} \\
\midrule
Conv1 & $112\times112$ & 64 & -- & 2 \\
MaxPool & $56\times56$ & 64 & -- & 2 \\
Layer1 & $56\times56$ & 64 & 3 & 1 \\
Layer2 & $28\times28$ & 128 & 4 & 2 \\
Layer3 & $14\times14$ & 256 & 6 & 2 \\
Layer4 & $7\times7$ & 512 & 3 & 2 \\
AvgPool + FC & -- & 15 & -- & -- \\
\bottomrule
\end{tabular}
\end{table}

Total: $\sim$21.3M parameters (much less than CNN's 95.6M).

\section{Experimental Results}

\begin{table}[H]
\centering
\caption{ResNet-34 Results (small-scale, 60 training images)}
\begin{tabular}{lccc}
\toprule
\textbf{Metric} & \textbf{Train} & \textbf{Val} & \textbf{Test} \\
\midrule
Loss & 0.3065 & 0.3742 & -- \\
AUC & 0.7342 & 0.4462 & \textbf{0.4462} \\
\bottomrule
\end{tabular}
\end{table}

\begin{figure}[H]
\centering
\includegraphics[width=0.8\textwidth]{../assets/image/output-resnet.png}
\caption{ResNet-34 training history}
\label{fig:resnet}
\end{figure}

\begin{tcolorbox}[colback=yellow!5!white,colframe=orange!75!black,title=Analysis]
ResNet-34 trained from scratch on small data yields poor results (AUC 0.4462). Deep architectures require pretrained weights or large-scale data to be effective.
\end{tcolorbox}

% ============================================================
% CHAPTER 5: VISION TRANSFORMER
% ============================================================
\chapter{Vision Transformer (ViT)}
\label{chap:vit}

\section{Architecture Overview}
ViT converts image classification into a sequence problem:
\begin{enumerate}
  \item Split $224\times224$ image into $32\times32$ patches $\Rightarrow$ 49 patches.
  \item Linear projection to embedding dimension 64.
  \item Add positional embedding + [CLS] token.
  \item Pass through 8 Transformer Encoder blocks (4 attention heads).
  \item MLP Head: [CLS] output $\to$ 15 classes (sigmoid).
\end{enumerate}

Total: \textbf{9,005,839 parameters}.

\section{ViT-v1: Training from Scratch}

Configuration: Adam optimizer, lr = $10^{-4}$, weight decay = $10^{-6}$, 10 epochs.

\begin{table}[H]
\centering
\caption{ViT-v1 Results (small-scale)}
\begin{tabular}{lccc}
\toprule
\textbf{Metric} & \textbf{Train} & \textbf{Val} & \textbf{Test} \\
\midrule
Loss & 0.2569 & 0.2717 & 0.2534 \\
Accuracy & 90.56\% & 90.00\% & \textbf{91.33\%} \\
AUC & 0.5883 & 0.5868 & \textbf{0.5854} \\
\bottomrule
\end{tabular}
\end{table}

\begin{figure}[H]
\centering
\includegraphics[width=0.8\textwidth]{../assets/image/output-roc-vit-v1.png}
\caption{ViT-v1 per-class ROC curves}
\label{fig:vit_v1_roc}
\end{figure}

\section{ViT-v2: Improved with SGD and Early Stopping}

Changes from v1: SGD optimizer, weight decay = $10^{-5}$, early stopping patience = 3.

\begin{table}[H]
\centering
\caption{ViT-v2 Results (small-scale)}
\begin{tabular}{lccc}
\toprule
\textbf{Metric} & \textbf{Train} & \textbf{Val} & \textbf{Test} \\
\midrule
Loss & 0.2657 & 0.2413 & 0.2749 \\
Accuracy & 89.78\% & 91.67\% & \textbf{89.67\%} \\
AUC & 0.5630 & 0.5947 & \textbf{0.6303} \\
\bottomrule
\end{tabular}
\end{table}

\begin{figure}[H]
\centering
\includegraphics[width=0.8\textwidth]{../assets/image/output-roc-vit-v2.png}
\caption{ViT-v2 per-class ROC curves}
\label{fig:vit_v2_roc}
\end{figure}

\section{ViT-ResNet: Pretrained (timm vit\_base\_patch16\_224)}

Using \texttt{vit\_base\_patch16\_224} pretrained on ImageNet. Total: \textbf{85,810,191 parameters}.

\begin{table}[H]
\centering
\caption{ViT-ResNet Pretrained Results (small-scale)}
\begin{tabular}{lccc}
\toprule
\textbf{Metric} & \textbf{Train} & \textbf{Val} & \textbf{Test} \\
\midrule
Loss & 0.2425 & 0.3232 & 0.3768 \\
Accuracy & 90.22\% & 86.67\% & \textbf{87.00\%} \\
AUC & 0.8820 & 0.6673 & \textbf{0.6694} \\
\bottomrule
\end{tabular}
\end{table}

\begin{figure}[H]
\centering
\includegraphics[width=0.8\textwidth]{../assets/image/output-vit-resnet.png}
\caption{ViT-ResNet pretrained training history}
\label{fig:vit_resnet}
\end{figure}

% ============================================================
% CHAPTER 6: FULL-SCALE EXPERIMENT
% ============================================================
\chapter{Full-scale Experiment: ViT on 112,120 Images}
\label{chap:fullscale}

\section{Configuration}
\begin{itemize}
  \item \textbf{Dataset:} 112,120 images, split by Patient ID
  \item \textbf{Split:} Train 78,614 / Val 11,212 / Test 22,294
  \item \textbf{Patients:} 21,563 / 3,081 / 6,161
  \item \textbf{Model:} ViT (patch 32, proj\_dim 64, 4 heads, 8 layers)
  \item \textbf{Epochs:} 10, Batch size: 32
\end{itemize}

\section{Results}

\begin{table}[H]
\centering
\caption{ViT Full-scale Results (112K images)}
\begin{tabular}{lccc}
\toprule
\textbf{Metric} & \textbf{Train} & \textbf{Val} & \textbf{Test} \\
\midrule
Loss & 0.1985 & 0.1967 & \textbf{0.2001} \\
Accuracy & 92.93\% & 93.01\% & \textbf{92.91\%} \\
AUC (Macro) & 0.7195 & 0.7264 & \textbf{0.7225} \\
\bottomrule
\end{tabular}
\end{table}

\section{Per-class AUC}

\begin{table}[H]
\centering
\caption{Per-class Test AUC (ViT Full-scale)}
\label{tab:perclass_auc}
\begin{tabular}{lclc}
\toprule
\textbf{Disease} & \textbf{AUC} & \textbf{Disease} & \textbf{AUC} \\
\midrule
Edema & \textbf{0.8422} & Pleural\_Thick. & 0.6997 \\
Cardiomegaly & 0.7996 & Fibrosis & 0.6977 \\
Effusion & 0.7880 & Mass & 0.6762 \\
Consolidation & 0.7615 & Pneumonia & 0.6710 \\
Pneumothorax & 0.7540 & Infiltration & 0.6614 \\
Hernia & 0.7460 & Nodule & 0.5747 \\
Emphysema & 0.7375 & & \\
Atelectasis & 0.7170 & \textbf{Macro Avg} & \textbf{0.7225} \\
No Finding & 0.7114 & & \\
\bottomrule
\end{tabular}
\end{table}

% ============================================================
% CHAPTER 7: COMPARISON & ANALYSIS
% ============================================================
\chapter{Comparison and Analysis}
\label{chap:comparison}

\section{Summary Comparison}

\begin{table}[H]
\centering
\caption{All Models Comparison}
\label{tab:comparison}
\begin{tabular}{lcccccc}
\toprule
\textbf{Model} & \textbf{Data} & \textbf{Params} & \textbf{Test AUC} & \textbf{Test Acc} & \textbf{Test Loss} & \textbf{Epochs} \\
\midrule
CNN & 60 & 95.6M & 0.5777 & -- & -- & 10 \\
ResNet-34 & 60 & 21.3M & 0.4462 & -- & -- & 10 \\
ViT-v1 & 60 & 9.0M & 0.5854 & 91.33\% & 0.2534 & 10 \\
ViT-v2 & 60 & 9.0M & 0.6303 & 89.67\% & 0.2749 & 9 \\
ViT-ResNet & 60 & 85.8M & 0.6694 & 87.00\% & 0.3768 & 10 \\
\midrule
\textbf{ViT Final} & \textbf{112K} & \textbf{9.0M} & \textbf{0.7225} & \textbf{92.91\%} & \textbf{0.2001} & \textbf{10} \\
\bottomrule
\end{tabular}
\end{table}

\section{Key Findings}

\begin{enumerate}
  \item \textbf{Data scale is decisive:} Same ViT architecture, AUC increased from 0.5854 (60 images) to \textbf{0.7225} (112K images) --- a 23.4\% improvement.

  \item \textbf{Transfer learning is effective:} Pretrained ViT-ResNet achieved AUC 0.6694 with only 60 images, outperforming all from-scratch models on small data.

  \item \textbf{CNN is unsuitable for limited data:} 99\% of parameters in the FC layer leads to severe overfitting (train AUC 0.91 vs test 0.58).

  \item \textbf{ResNet needs large data:} Training from scratch on 60 images yields the lowest AUC (0.4462).

  \item \textbf{ViT-v2 improves over v1:} SGD + early stopping increases AUC from 0.5854 to 0.6303.
\end{enumerate}

% ============================================================
% CHAPTER 8: CONCLUSION
% ============================================================
\chapter{Conclusion and Future Work}
\label{chap:conclusion}

\section{Conclusions}
\begin{enumerate}
  \item Vision Transformer achieves the best performance on the full dataset (AUC 0.7225, Acc 92.91\%).
  \item Transfer learning is critical when data is limited.
  \item Data scale matters more than model architecture.
  \item Simple CNN suffers from overfitting due to oversized FC layers.
\end{enumerate}

\section{Future Directions}
\begin{itemize}
  \item Apply Focal Loss / Asymmetric Loss to handle class imbalance.
  \item Experiment with Swin Transformer for multi-scale features.
  \item Build ensemble models for improved performance.
  \item Increase training epochs and use learning rate scheduling.
  \item Evaluate on CheXpert and MIMIC-CXR to verify generalizability.
\end{itemize}

% ============================================================
% CHAPTER 9: TRAINING CONFIGURATION
% ============================================================
\chapter{Training Configuration}
\label{chap:config}

\begin{table}[H]
\centering
\caption{Common Training Configuration}
\begin{tabular}{ll}
\toprule
\textbf{Parameter} & \textbf{Value} \\
\midrule
Image size & $224 \times 224$ \\
Batch size & 32 (16 for ViT-ResNet) \\
Number of classes & 15 \\
Loss function & BCEWithLogitsLoss \\
Learning rate & $1 \times 10^{-4}$ \\
Weight decay & $1 \times 10^{-6}$ ($10^{-5}$ for ViT-v2) \\
Optimizer & AdamW (SGD for ViT-v2) \\
Epochs & 10 \\
GPU & NVIDIA GeForce RTX 3060 Laptop (6.4 GB) \\
CUDA & 12.6 \\
Framework & PyTorch 2.x \\
\bottomrule
\end{tabular}
\end{table}

% ============================================================
% REFERENCES
% ============================================================
\begin{thebibliography}{20}

\bibitem{jain2024comparative}
Jain, M. et al. (2024).
\textit{A Comparative Study of CNN, ResNet, and Vision Transformers for Multi-Classification of Chest Diseases}.
arXiv:2406.00237.

\bibitem{wang2017chestxray14}
Wang, X. et al. (2017).
\textit{ChestX-ray8: Hospital-scale Chest X-ray Database and Benchmarks}.
CVPR.

\bibitem{dosovitskiy2020vit}
Dosovitskiy, A. et al. (2020).
\textit{An Image is Worth 16x16 Words: Transformers for Image Recognition at Scale}.
ICLR.

\bibitem{he2016resnet}
He, K. et al. (2016).
\textit{Deep Residual Learning for Image Recognition}.
CVPR.

\bibitem{rajpurkar2017chexnet}
Rajpurkar, P. et al. (2017).
\textit{CheXNet: Radiologist-Level Pneumonia Detection on Chest X-Rays with Deep Learning}.
arXiv:1711.05225.

\bibitem{lin2017focal}
Lin, T.-Y. et al. (2017).
\textit{Focal Loss for Dense Object Detection}.
ICCV.

\bibitem{ridnik2021asymmetric}
Ridnik, T. et al. (2021).
\textit{Asymmetric Loss For Multi-Label Classification}.
ICCV.

\bibitem{vaswani2017attention}
Vaswani, A. et al. (2017).
\textit{Attention Is All You Need}.
NeurIPS.

\bibitem{liu2021swin}
Liu, Z. et al. (2021).
\textit{Swin Transformer: Hierarchical Vision Transformer using Shifted Windows}.
ICCV.

\bibitem{irvin2019chexpert}
Irvin, J. et al. (2019).
\textit{CheXpert: A Large Chest Radiograph Dataset with Uncertainty Labels and Expert Comparison}.
AAAI.

\end{thebibliography}

\end{document}
