% ============================================================
% CHAPTER 3: DATASET - PHÂN TÍCH NIH CHEST X-RAY
% ============================================================
\chapter{Phân tích Dataset: NIH Chest X-ray}

\section{Tổng quan Dataset}

\paperref{Section 4.1 - Dataset}

\subsection{Thông tin cơ bản}

\begin{tcolorbox}[colback=blue!5!white,colframe=blue!75!black,title=NIH Chest X-ray Dataset]
\begin{itemize}
    \item \textbf{Tên đầy đủ:} ChestX-ray14 (NIH Clinical Center)
    \item \textbf{Số lượng ảnh:} 112,120 frontal-view X-ray images
    \item \textbf{Số bệnh nhân:} 30,805 unique patients
    \item \textbf{Số nhãn bệnh:} 14 bệnh + 1 ``No Finding'' = 15 classes
    \item \textbf{Nguồn nhãn:} Text-mining từ radiology reports (weak labels)
    \item \textbf{Kích thước ảnh gốc:} 1024 × 1024 pixels
\end{itemize}
\end{tcolorbox}

\subsection{Trích dẫn từ Paper}

\begin{tcolorbox}[colback=yellow!5!white,colframe=yellow!75!black,title=Paper Quote - Dataset]
\textit{``To evaluate the performance of our model architectures, we utilized two freely available datasets: the NIH Chest X-ray dataset comprising 112,120 X-ray images with disease labels from 30,805 unique patients, and a Random Sample of the NIH Chest X-ray Dataset, containing 5,606 X-ray images. Both datasets involved multi-class classification across 15 classes, each representing different disease labels.''}
\end{tcolorbox}

\section{Danh sách 15 Classes}

\subsection{Phân loại bệnh lý}

\begin{table}[H]
\centering
\caption{15 Classes trong NIH Chest X-ray Dataset}
\begin{tabular}{clcp{6cm}}
\toprule
\textbf{ID} & \textbf{Tên bệnh} & \textbf{Tỷ lệ (\%)} & \textbf{Mô tả} \\
\midrule
0 & Cardiomegaly & 2.48 & Tim to \\
1 & Emphysema & 2.24 & Khí phế thũng \\
2 & Effusion & 11.88 & Tràn dịch màng phổi \\
3 & Hernia & 0.20 & Thoát vị \\
4 & Nodule & 5.65 & Nốt phổi \\
5 & Pneumothorax & 4.73 & Tràn khí màng phổi \\
6 & Atelectasis & 10.31 & Xẹp phổi \\
7 & Pleural\_Thickening & 3.02 & Dày màng phổi \\
8 & Mass & 5.16 & Khối u \\
9 & Edema & 2.05 & Phù phổi \\
10 & Consolidation & 4.16 & Đông đặc phổi \\
11 & Infiltration & 17.74 & Thâm nhiễm \\
12 & Fibrosis & 1.50 & Xơ phổi \\
13 & Pneumonia & 1.28 & Viêm phổi \\
14 & No Finding & 53.84 & Không phát hiện bệnh \\
\bottomrule
\end{tabular}
\end{table}

\subsection{Phân tích mất cân bằng lớp (Class Imbalance)}

\begin{tcolorbox}[colback=red!5!white,colframe=red!75!black,title=Vấn đề Class Imbalance]
\textbf{Quan sát quan trọng:}
\begin{itemize}
    \item ``No Finding'' chiếm \textbf{53.84\%} - hơn một nửa dataset
    \item ``Hernia'' chỉ chiếm \textbf{0.20\%} - rất hiếm
    \item Tỷ lệ cao nhất / thấp nhất = 53.84 / 0.20 = \textbf{269 lần}
\end{itemize}

\textbf{Hậu quả:}
\begin{itemize}
    \item Model có thể thiên về dự đoán ``No Finding''
    \item Accuracy cao nhưng chưa chắc đã detect tốt bệnh hiếm
    \item Cần metrics như AUC thay vì chỉ accuracy
\end{itemize}
\end{tcolorbox}

\section{Bản chất Multi-label}

\subsection{Multi-label vs Multi-class}

\begin{table}[H]
\centering
\caption{So sánh Multi-class và Multi-label Classification}
\begin{tabular}{lp{5cm}p{5cm}}
\toprule
\textbf{Aspect} & \textbf{Multi-class} & \textbf{Multi-label} \\
\midrule
Số nhãn/sample & Chính xác 1 & Có thể nhiều (0, 1, 2, ...) \\
Output activation & Softmax & Sigmoid (independent) \\
Loss function & Categorical CE & Binary CE \\
Ví dụ & Cat OR Dog OR Bird & Cat AND Dog (có thể cả hai) \\
NIH Dataset & Không phù hợp & \checkmark Phù hợp \\
\bottomrule
\end{tabular}
\end{table}

\subsection{Ví dụ Multi-label trong NIH}

Một ảnh X-quang có thể có nhiều bệnh đồng thời:

\begin{lstlisting}[caption={Ví dụ multi-label trong dataset}]
# Ảnh 00000001_000.png có thể có labels:
labels = [0, 0, 1, 0, 0, 0, 1, 0, 0, 0, 0, 0, 0, 0, 0]
#         ^     ^        ^
#         |     |        |
#         |     |        Atelectasis (index 6)
#         |     Effusion (index 2)
#         Cardiomegaly (index 0) - Không có

# Bệnh nhân có: Effusion + Atelectasis (2 bệnh đồng thời)
\end{lstlisting}

\section{Data Pipeline trong Code}

\coderef{data.ipynb}

\subsection{Loading và Preprocessing}

\begin{lstlisting}[caption={Data loading từ data.ipynb (PyTorch version)}]
class ChestXrayDataset(Dataset):
    def __init__(self, dataframe, images_path, labels, transform=None):
        self.dataframe = dataframe.reset_index(drop=True)
        self.images_path = images_path
        self.labels = labels
        self.transform = transform
        
    def __len__(self):
        return len(self.dataframe)
    
    def __getitem__(self, idx):
        # Load image
        img_name = self.dataframe.iloc[idx]['Image Index']
        img_path = os.path.join(self.images_path, img_name)
        image = Image.open(img_path).convert('RGB')
        
        # Apply transforms
        if self.transform:
            image = self.transform(image)
        
        # Get labels as one-hot vector
        label = torch.tensor(
            self.dataframe.iloc[idx][self.labels].values.astype(float),
            dtype=torch.float32
        )
        
        return image, label
\end{lstlisting}

\subsection{Data Augmentation}

\paperref{Section 4.2 - Models}

\begin{tcolorbox}[colback=green!5!white,colframe=green!75!black,title=Augmentation từ Paper]
\textit{``We also performed various data augmentations on both datasets. For the Chest X-ray dataset, we applied resizing, random horizontal flip, and random rotation.''}
\end{tcolorbox}

\begin{lstlisting}[caption={Data augmentation transforms}]
train_transform = transforms.Compose([
    transforms.Resize((224, 224)),        # Resize to standard size
    transforms.RandomHorizontalFlip(p=0.5), # Random flip
    transforms.RandomRotation(degrees=5),   # Small rotation
    transforms.ColorJitter(brightness=0.1, contrast=0.1),
    transforms.ToTensor(),
    transforms.Normalize(
        mean=[0.485, 0.456, 0.406],  # ImageNet stats
        std=[0.229, 0.224, 0.225]
    )
])

val_transform = transforms.Compose([
    transforms.Resize((224, 224)),
    transforms.ToTensor(),
    transforms.Normalize(
        mean=[0.485, 0.456, 0.406],
        std=[0.229, 0.224, 0.225]
    )
])
\end{lstlisting}

\subsection{Lưu ý về Horizontal Flip trong X-ray}

\begin{tcolorbox}[colback=orange!5!white,colframe=orange!75!black,title=Cảnh báo: Horizontal Flip]
Trong X-quang, flip ngang có thể gây vấn đề:
\begin{itemize}
    \item Tim thường nằm bên trái → flip làm tim nằm bên phải (Dextrocardia - bất thường)
    \item Một số bệnh có tính ``laterality'' (bên phải/trái khác nhau)
\end{itemize}

\textbf{Khuyến nghị:} Cần làm ablation study để đánh giá ảnh hưởng của flip.
\end{tcolorbox}

\section{Data Split}

\subsection{Paper Description}

\paperref{Section 4.1 - Dataset}

\begin{tcolorbox}[colback=yellow!5!white,colframe=yellow!75!black,title=Data Split từ Paper]
\textit{``However, our model training was conducted on a subset of 85,000 images from this Random Sample Dataset. We observed a faster convergence to optimal outputs within this subset.''}
\end{tcolorbox}

\subsection{Implementation trong Code}

\begin{lstlisting}[caption={Data split implementation}]
from sklearn.model_selection import train_test_split

# Standard split: 60% train, 20% val, 20% test
train_df, temp_df = train_test_split(
    full_df, 
    test_size=0.4, 
    random_state=42
)
val_df, test_df = train_test_split(
    temp_df, 
    test_size=0.5, 
    random_state=42
)

print(f"Training samples: {len(train_df)}")
print(f"Validation samples: {len(val_df)}")
print(f"Test samples: {len(test_df)}")
\end{lstlisting}

\subsection{Vấn đề Data Leakage}

\begin{tcolorbox}[colback=red!5!white,colframe=red!75!black,title=Cảnh báo: Patient-level Split]
\textbf{Vấn đề:} Paper không đề cập rõ về patient-level split.

\textbf{Rủi ro:} Nếu split theo image (không theo patient):
\begin{itemize}
    \item Cùng một bệnh nhân có thể có nhiều ảnh
    \item Ảnh của cùng bệnh nhân có thể nằm ở cả train và test
    \item Model có thể ``nhớ'' bệnh nhân thay vì học features bệnh
    \item Kết quả đánh giá bị inflate (cao giả tạo)
\end{itemize}

\textbf{Khuyến nghị:} Split theo Patient ID để đảm bảo generalization thực sự.
\end{tcolorbox}

\section{Thống kê Dataset sử dụng trong thực nghiệm}

\begin{table}[H]
\centering
\caption{Dataset splits trong các thực nghiệm}
\begin{tabular}{lccc}
\toprule
\textbf{Experiment} & \textbf{Train} & \textbf{Val} & \textbf{Test} \\
\midrule
Paper (Full) & 68,000 & 8,500 & 8,500 \\
Paper (Subset) & 3,363 & 1,121 & 1,122 \\
Our experiment (demo) & 60 & 20 & 20 \\
\bottomrule
\end{tabular}
\end{table}

\textbf{Lưu ý:} Thực nghiệm demo của chúng tôi sử dụng dataset rất nhỏ (100 samples) để test pipeline. Kết quả chưa đáng tin cậy để đánh giá model.
