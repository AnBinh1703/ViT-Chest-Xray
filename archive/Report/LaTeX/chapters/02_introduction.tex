% ============================================================
% CHAPTER 2: INTRODUCTION - PHÂN TÍCH BỐI CẢNH VÀ ĐỘNG LỰC
% ============================================================
\chapter{Giới thiệu và Bối cảnh Nghiên cứu}

\section{Bối cảnh lâm sàng}

\subsection{Vai trò của X-quang ngực trong chẩn đoán}

\paperref{Section 1 - Introduction}

X-quang ngực (Chest X-ray) là một trong những phương pháp chẩn đoán hình ảnh phổ biến nhất trong y tế:

\begin{itemize}
    \item \textbf{Chi phí thấp:} So với CT, MRI, X-quang có chi phí thấp hơn đáng kể
    \item \textbf{Thời gian nhanh:} Kết quả có thể có trong vài phút
    \item \textbf{Sàng lọc ban đầu:} Thường là xét nghiệm đầu tiên khi nghi ngờ bệnh phổi
    \item \textbf{Khả năng phát hiện:} Có thể phát hiện nhiều bệnh lý: viêm phổi, xẹp phổi, tràn dịch, tim to, v.v.
\end{itemize}

\subsection{Thách thức trong chẩn đoán}

Paper nhấn mạnh các thách thức chính:

\begin{tcolorbox}[colback=yellow!5!white,colframe=yellow!75!black,title=Trích dẫn từ Paper (Section 1)]
\textit{``Although chest X-ray imaging is a relatively low cost tool for diagnosis, radiologists are needed to analyze these images. However the limited access to radiologists in many areas, and the variability between radiologists can be a problem.''}
\end{tcolorbox}

\begin{enumerate}
    \item \textbf{Thiếu bác sĩ X-quang:} Đặc biệt ở vùng sâu, vùng xa
    \item \textbf{Biến thiên giữa người đọc:} Inter-observer variability - hai bác sĩ có thể đưa ra chẩn đoán khác nhau
    \item \textbf{Khối lượng dữ liệu lớn:} Một bác sĩ có thể phải đọc hàng trăm ảnh mỗi ngày
    \item \textbf{Dấu hiệu tinh vi:} Một số bệnh có dấu hiệu rất khó nhận biết
\end{enumerate}

\section{Động lực nghiên cứu AI trong chẩn đoán hình ảnh}

\subsection{Lợi ích của Machine Learning}

\paperref{Section 1 - Introduction}

\begin{tcolorbox}[colback=blue!5!white,colframe=blue!75!black,title=Lợi ích của ML trong Medical Imaging]
\begin{enumerate}
    \item \textbf{Tăng độ chính xác:} Models có thể học các patterns phức tạp từ dữ liệu lớn
    \item \textbf{Mở rộng khả năng tiếp cận:} Giúp vùng thiếu bác sĩ có công cụ hỗ trợ chẩn đoán
    \item \textbf{Nhất quán:} Không bị ảnh hưởng bởi mệt mỏi hay bias cá nhân
    \item \textbf{Phát hiện chi tiết:} Có thể phát hiện các dấu hiệu mà mắt người bỏ qua
\end{enumerate}
\end{tcolorbox}

\subsection{Tiến hóa của Deep Learning trong Medical Imaging}

\begin{figure}[H]
\centering
\begin{tikzpicture}[node distance=2cm, auto]
% This is a conceptual representation
\end{tikzpicture}
\caption{Timeline phát triển của Deep Learning trong Medical Imaging}
\end{figure}

\begin{enumerate}
    \item \textbf{2012 - AlexNet:} CNN đầu tiên thắng ImageNet, mở ra kỷ nguyên Deep Learning
    \item \textbf{2015 - ResNet:} Giải quyết vanishing gradient, cho phép train mạng rất sâu
    \item \textbf{2017 - Attention mechanisms:} Transformer ra đời cho NLP
    \item \textbf{2020 - Vision Transformer (ViT):} Áp dụng Transformer cho Computer Vision
    \item \textbf{2021+:} ViT và variants trở thành state-of-the-art trong nhiều tasks
\end{enumerate}

\section{Câu hỏi nghiên cứu của Paper}

\paperref{Section 1 - Introduction}

Paper đặt ra các câu hỏi nghiên cứu cốt lõi:

\begin{tcolorbox}[colback=green!5!white,colframe=green!75!black,title=Research Questions]
\begin{enumerate}
    \item[\textbf{RQ1:}] CNN, ResNet và ViT khác nhau như thế nào về hiệu năng trong phân loại bệnh X-quang ngực?
    \item[\textbf{RQ2:}] ViT trained from scratch có thể cạnh tranh với ResNet không?
    \item[\textbf{RQ3:}] Pre-training trên ImageNet có giúp ViT cải thiện đáng kể không?
    \item[\textbf{RQ4:}] Attention maps có thể cung cấp insights hữu ích cho việc diễn giải không?
\end{enumerate}
\end{tcolorbox}

\section{Đóng góp của Paper}

\subsection{Đóng góp kỹ thuật}

\begin{enumerate}
    \item \textbf{So sánh toàn diện:} Đánh giá 5 models (CNN, ResNet, ViT-v1, ViT-v2, ViT-ResNet) trên cùng một dataset
    \item \textbf{Implementation chi tiết:} Cung cấp code và hướng dẫn tái lập
    \item \textbf{Hyperparameter tuning:} Thử nghiệm nhiều cấu hình khác nhau
    \item \textbf{Interpretability:} Trích xuất và phân tích attention maps
\end{enumerate}

\subsection{Đóng góp thực tiễn}

\begin{enumerate}
    \item \textbf{Guidance cho practitioners:} Khi nào nên dùng CNN, ResNet, hay ViT
    \item \textbf{Insights về pre-training:} Tầm quan trọng của transfer learning trong medical imaging
    \item \textbf{Open-source code:} Repository công khai để cộng đồng sử dụng
\end{enumerate}

\section{Cấu trúc Paper và Báo cáo}

\subsection{Cấu trúc Paper gốc}

\begin{table}[H]
\centering
\caption{Cấu trúc Paper và nội dung tương ứng}
\begin{tabular}{clp{8cm}}
\toprule
\textbf{Section} & \textbf{Tên} & \textbf{Nội dung chính} \\
\midrule
1 & Introduction & Bối cảnh, động lực, đóng góp \\
2 & Related Work & Các nghiên cứu liên quan \\
3 & Approach & Mô tả kiến trúc CNN, ResNet, ViT \\
4 & Experiment & Dataset, training, evaluation \\
5 & Discussion & Phân tích kết quả, attention maps \\
6 & Conclusion & Kết luận và hướng phát triển \\
\bottomrule
\end{tabular}
\end{table}

\subsection{Mapping với báo cáo này}

Báo cáo này mở rộng và phân tích sâu hơn:

\begin{itemize}
    \item \textbf{Chapter 3:} Dataset - Phân tích chi tiết NIH Chest X-ray
    \item \textbf{Chapter 4:} CNN - Lý thuyết + Code mapping
    \item \textbf{Chapter 5:} ResNet - Lý thuyết + Code mapping
    \item \textbf{Chapter 6:} ViT - Lý thuyết + Code mapping (trọng tâm)
    \item \textbf{Chapter 7:} Experiments - Kết quả và phân tích
    \item \textbf{Chapter 8:} Implementation - Quá trình tái lập và cải tiến
    \item \textbf{Chapter 9:} Conclusion - Kết luận và đề xuất
\end{itemize}
