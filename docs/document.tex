% ======================================================================
% SECTION X: BÁO CÁO RÀ SOÁT NGHIÊN CỨU (RESEARCH AUDIT REPORT)
% ======================================================================
\section{Báo cáo rà soát nghiên cứu cho mô hình ViT-Chest-Xray}
\label{sec:audit}

\subsection*{Thông tin chung}

\begin{itemize}[noitemsep]
	\item \textbf{Người rà soát:} Senior Research Engineer \& Scientific Reviewer
	\item \textbf{Ngày rà soát:} 04/02/2026
	\item \textbf{Dự án:} ViT-Chest-Xray Multi-Label Classification (Đồ án cuối kỳ môn Deep Learning -- Nhóm 1)
	\item \textbf{Bài báo tham chiếu:} Jain et al. (2024), arXiv:2406.00237
\end{itemize}

\subsection*{Tóm tắt điều hành}

Báo cáo này đánh giá mức độ ``chuẩn nghiên cứu'' (research-grade) của dự án ViT-Chest-Xray, vốn là bản cài đặt lại bằng PyTorch của bài báo \cite{jain2024comparative}, so sánh các kiến trúc CNN, ResNet và Vision Transformer cho bài toán phân loại đa nhãn (multi-label) trên bộ dữ liệu NIH ChestX-ray14.

\begin{table}[H]
	\centering
	\caption{Tóm tắt các chỉ số đánh giá chất lượng dự án sau rà soát}
	\label{tab:audit_summary}
	\begin{tabular}{ll}
		\toprule
		\textbf{Tiêu chí} & \textbf{Trạng thái} \\
		\midrule
		Best Test AUC & 0.7225 (ViT huấn luyện từ đầu) \\
		Best Validation AUC & 0.7272 \\
		Tính đúng đắn khoa học & \checkmark\ Thiết lập multi-label chuẩn (sigmoid + BCE) \\
		Nguy cơ rò rỉ dữ liệu & Đã giảm thiểu với chia \textit{theo bệnh nhân} \\
		Khả năng tái lập (reproducibility) & \checkmark\ Cố định seed, cấu hình ghi rõ ràng \\
		Mức độ sẵn sàng & \textbf{85\%} -- Sẵn sàng nộp, chỉ cần tinh chỉnh nhỏ \\
		\bottomrule
	\end{tabular}
\end{table}

% ----------------------------------------------------------------------
\subsection{Bước 1: Rà soát tổng thể dự án}

\subsubsection{Bản đồ dự án (Project Map)}

\begin{table}[H]
	\centering
	\caption{Tổng quan các file/notebook chính và đánh giá sơ bộ}
	\label{tab:audit_files}
	\begin{tabular}{p{3.0cm}p{3.3cm}p{2.9cm}p{3.2cm}p{1.1cm}}
		\toprule
		\textbf{File/Notebook} & \textbf{Mục đích} & \textbf{Vấn đề/Nguy cơ} & \textbf{Khuyến nghị chính} & \textbf{Điểm} \\
		\midrule
		\texttt{config.py} & Cấu hình trung tâm (đường dẫn, hyperparameters) &
		Đường dẫn Windows hard-code, khó port & Dùng \texttt{pathlib} + biến môi trường & 6/10 \\
		\texttt{data\_download.ipynb} & Tải dữ liệu NIH từ Kaggle & Không có tiến trình, không kiểm tra checksum &
		Thêm progress bar, kiểm tra toàn vẹn dữ liệu & 5/10 \\
		\texttt{data.ipynb} & Load, tiền xử lý, DataLoader & \textbf{Chia theo ảnh} $\Rightarrow$ rò rỉ dữ liệu & Chuyển sang chia \textbf{theo bệnh nhân} & 5/10 \\
		\texttt{cnn.ipynb} & CNN baseline & 95M tham số, Flatten quá lớn, overfit & Dùng Global Avg Pooling, giảm FC & 4/10 \\
		\texttt{resnet.ipynb} & ResNet-34 tự cài & Không dùng pretrain ImageNet & Dùng \texttt{torchvision} pretrained & 7/10 \\
		\texttt{ViT-v1.ipynb} & ViT từ đầu (phiên bản cơ bản) & Thiếu scheduler, early stopping & Thêm ReduceLROnPlateau, early stop & 6/10 \\
		\texttt{ViT-v2.ipynb} & ViT từ đầu (có tối ưu train) & Early stop theo loss, chưa theo AUC & Theo dõi AUC val cho early stop & 7/10 \\
		\texttt{ViT-ResNet.ipynb} & ViT pretrained từ \texttt{timm} & Dùng ViT-Base 86M tham số, tốn RAM & Cân nhắc ViT-Small, gradient checkpointing & 8/10 \\
		\textbf{\texttt{Final\_ViT\_ChestXray.ipynb}} & \textbf{Notebook tổng hợp cuối} &
		Đã gom pipeline hoàn chỉnh & \checkmark\ Gần mức production-ready & \textbf{9/10} \\
		\texttt{improve/} & Thử nghiệm loss nâng cao & Phân tán, nhiều notebook nhỏ & Đã được tổng hợp ý chính vào Final notebook & 8/10 \\
		\texttt{Report/Group1\_Deeplearning.tex} & Báo cáo LaTeX & Còn một số chi tiết có thể đào sâu & Cập nhật số liệu mới, thêm bảng/giải thích & 8/10 \\
		\bottomrule
	\end{tabular}
\end{table}

\subsubsection{Vấn đề nghiêm trọng: Rò rỉ dữ liệu do chia theo ảnh}

\textbf{Mô tả vấn đề:} Bản cài đặt ban đầu (và cả bài báo gốc) dùng chiến lược chia train/validation/test \textit{theo ảnh}. Vì mỗi bệnh nhân có thể có nhiều phim X-quang, việc chia theo ảnh khiến các phim của cùng một bệnh nhân có thể xuất hiện đồng thời ở cả tập huấn luyện và tập kiểm tra.

\textbf{Hệ quả:}
\begin{itemize}[noitemsep]
	\item Mô hình có thể học các đặc trưng riêng của bệnh nhân thay vì đặc trưng bệnh lý.
	\item Điểm AUC trên test bị thổi phồng (có thể cao hơn thực tế 5--15\%).
	\item Khả năng tổng quát sang bệnh nhân mới bị đánh giá sai lệch.
\end{itemize}

\textbf{Tình trạng hiện tại:} Đã được \textbf{khắc phục} trong \texttt{Final\_ViT\_ChestXray.ipynb} bằng cách chia theo \textit{ID bệnh nhân}:

\begin{lstlisting}[language=Python, caption=Chia dữ liệu theo bệnh nhân, label={lst:patient_split}]
	# Tên ảnh có dạng: 00000001_000.png
	df['Patient ID'] = df['Image Index'].apply(lambda x: x.split('_')[0])
	
	# Lấy danh sách bệnh nhân duy nhất
	unique_patients = df['Patient ID'].unique()
	
	# Chia train/test theo bệnh nhân
	train_patients, test_patients = train_test_split(
	unique_patients, test_size=0.2, random_state=42
	)
\end{lstlisting}

\subsubsection{Các vấn đề và khắc phục khác}

\paragraph{Augmentation lật ngang (Horizontal Flip).}
\begin{itemize}[noitemsep]
	\item Vấn đề: dùng xác suất lật ngang $p=0.5$ trên ảnh X-quang có thể đảo ngược vị trí giải phẫu (tim, phổi trái/phải).
	\item Khắc phục: giảm xuống $p=0.3$, đồng thời ghi chú rõ trong phần phương pháp.
\end{itemize}

\paragraph{NaN trong tính AUC.}
\begin{itemize}[noitemsep]
	\item Vấn đề: Nếu một lớp chỉ toàn 0 hoặc toàn 1 trong một batch, AUC không được định nghĩa $\Rightarrow$ \texttt{roc\_auc\_score} có thể trả về NaN.
	\item Khắc phục: chỉ tính AUC trên những lớp có cả nhãn 0 và 1.
\end{itemize}

\begin{lstlisting}[language=Python, caption=Xử lý lớp không đủ 0/1 khi tính AUC, label={lst:auc_valid}]
	valid_classes = []
	for i in range(num_classes):
	if len(np.unique(targets[:, i])) > 1:  # cần cả 0 và 1
	valid_classes.append(i)
	
	if len(valid_classes) > 0:
	macro_auc = roc_auc_score(
	targets[:, valid_classes],
	outputs[:, valid_classes],
	average='macro'
	)
\end{lstlisting}

\paragraph{Hiệu quả kiến trúc CNN.}
\begin{itemize}[noitemsep]
	\item Vấn đề: lớp \texttt{Linear} sau Flatten có khoảng 95M tham số, gây lãng phí và overfit.
	\item Khuyến nghị: thay Flatten bằng Global Average Pooling để giảm số tham số xuống nhiều lần.
\end{itemize}

% ----------------------------------------------------------------------
\subsection{Bước 2: Phân tích notebook cuối \texttt{Final\_ViT\_ChestXray.ipynb}}

\subsubsection{Vai trò của notebook cuối}

Notebook \texttt{Final\_ViT\_ChestXray.ipynb} đóng vai trò:
\begin{itemize}[noitemsep]
	\item Gom toàn bộ pipeline (cấu hình, chia dữ liệu theo bệnh nhân, định nghĩa mô hình, huấn luyện, đánh giá).
	\item Chuẩn hóa lại cách tính AUC, lưu checkpoint, lưu config.
	\item Chạy thí nghiệm trên \textbf{toàn bộ} bộ dữ liệu NIH ChestX-ray14, không chỉ trên subset nhỏ.
\end{itemize}

\subsubsection{Những điểm đã được tổng hợp/khắc phục trong notebook Final}

\begin{table}[H]
	\centering
	\caption{Tóm tắt các thành phần đã được chuẩn hoá trong \texttt{Final\_ViT\_ChestXray.ipynb}}
	\label{tab:final_notebook_components}
	\begin{tabular}{p{4.2cm}p{7.8cm}p{1.4cm}}
		\toprule
		\textbf{Thành phần} & \textbf{Mô tả} & \textbf{Trạng thái} \\
		\midrule
		Reproducibility & Cố định seed = 42, sử dụng các tuỳ chọn deterministic của PyTorch & \checkmark \\
		Cấu hình & Định nghĩa lớp \texttt{Config} chứa toàn bộ hyperparameters & \checkmark \\
		Chia theo bệnh nhân & Hàm \texttt{get\_patient\_level\_split()} & \checkmark \\
		Pipeline multi-label & Dùng \texttt{BCEWithLogitsLoss} + \texttt{sigmoid} & \checkmark \\
		Mô hình CNN & Lớp \texttt{CNNClassifier} & \checkmark \\
		ResNet-34 & Hàm \texttt{create\_resnet34()} & \checkmark \\
		ViT từ đầu & Lớp \texttt{VisionTransformer} (9M tham số) & \checkmark \\
		ViT pretrained & Dùng \texttt{timm.create\_model('vit\_base\_patch16\_224', pretrained=True)} & \checkmark \\
		Đánh giá & Tính Macro AUC, per-class AUC, accuracy & \checkmark \\
		Lưu trữ & Lưu checkpoint mô hình tốt nhất, lưu file cấu hình JSON & \checkmark \\
		\bottomrule
	\end{tabular}
\end{table}

\subsubsection{Bảng kết quả cuối cùng (Final Metrics)}

\begin{table}[H]
	\centering
	\caption{Hiệu năng cuối cùng của các mô hình (từ log của notebook Final)}
	\label{tab:final_metrics}
	\begin{tabular}{lccccc}
		\toprule
		\textbf{Mô hình} & \textbf{Tham số} & \textbf{Best Val AUC} & \textbf{Test AUC} & \textbf{Test Acc (\%)} & \textbf{Ghi chú} \\
		\midrule
		CNN Baseline & $\sim$95M & 0.5998 & $\sim$0.58 & $\sim$89.0 & Overfit mạnh \\
		ResNet-34 & $\sim$21M & 0.5293 & $\sim$0.53 & $\sim$91.0 & Chưa dùng pretrained \\
		ViT-v1 (từ đầu) & 9.0M & 0.6431 & 0.5854 & 91.33 & Bản ViT cơ bản \\
		ViT-v2 (từ đầu) & 9.0M & 0.5947 & 0.6303 & 89.67 & Có scheduler \\
		ViT-Pretrained & $\sim$86M & N/A & 0.6694 & 87.00 & ViT-Base từ ImageNet \\
		\textbf{ViT (từ đầu, Final)} & \textbf{9.0M} & \textbf{0.7272} & \textbf{0.7225} & \textbf{92.91} & \textbf{Full dataset + chia theo bệnh nhân} \\
		\bottomrule
	\end{tabular}
\end{table}

Kết quả quan trọng nhất là mô hình ViT huấn luyện từ đầu trên toàn bộ dữ liệu, với chia tập theo bệnh nhân, đạt \textbf{Test Macro AUC = 0.7225} cùng Test Accuracy = 92.91\%.

% ----------------------------------------------------------------------
\subsection{Bước 3: Đánh giá từng notebook chính}

Dưới đây là tóm tắt ngắn gọn cho từng notebook:

\subsubsection{\texttt{cnn.ipynb}}

\begin{itemize}[noitemsep]
	\item \textbf{Mục tiêu:} Xây dựng CNN baseline cho phân loại đa nhãn.
	\item \textbf{Kiến trúc:} 2 lớp conv + ReLU + MaxPool, sau đó Flatten và 2 lớp fully-connected lớn.
	\item \textbf{Vấn đề:} Flatten tạo ra $\approx 186{,}624$ đặc trưng, FC đầu tiên $\approx 95$M tham số $\Rightarrow$ mô hình cồng kềnh, overfit.
	\item \textbf{Kết quả:} Train AUC $\approx 0.90$, Val AUC $\approx 0.60$ (overfit nặng).
	\item \textbf{Kết luận:} \textbf{4/10} -- chấp nhận được làm baseline, nhưng không hiệu quả cho thực tế.
\end{itemize}

\subsubsection{\texttt{resnet.ipynb}}

\begin{itemize}[noitemsep]
	\item \textbf{Mục tiêu:} Cài đặt ResNet-34 từ đầu.
	\item \textbf{Kiến trúc:} Chuẩn ResNet-34 với BasicBlock, skip connection, cấu hình [3, 4, 6, 3] khối.
	\item \textbf{Vấn đề:} Không sử dụng weight pretrain từ ImageNet, nên hiệu năng bị hạn chế.
	\item \textbf{Kết quả:} Val AUC $\approx 0.53$.
	\item \textbf{Kết luận:} \textbf{7/10} -- kiến trúc đúng, nhưng chưa tận dụng transfer learning.
\end{itemize}

\subsubsection{\texttt{ViT-v1.ipynb}}

\begin{itemize}[noitemsep]
	\item \textbf{Mục tiêu:} ViT từ đầu, bản cơ bản.
	\item \textbf{Kiến trúc:} Patch size 32, embedding dim 64, 8 encoder blocks, 4 attention heads, MLP head.
	\item \textbf{Vấn đề:} Không có learning rate scheduler, không early stopping.
	\item \textbf{Kết quả:} Val AUC $\approx 0.64$, Test AUC $\approx 0.59$.
	\item \textbf{Kết luận:} \textbf{6/10} -- mô hình đúng, cần tối ưu thêm quá trình huấn luyện.
\end{itemize}

\subsubsection{\texttt{ViT-v2.ipynb}}

\begin{itemize}[noitemsep]
	\item \textbf{Mục tiêu:} Cải thiện huấn luyện ViT (scheduler + early stopping).
	\item \textbf{Điểm mới:} Dùng \texttt{ReduceLROnPlateau}, early stopping với patience cố định.
	\item \textbf{Vấn đề:} Theo dõi early stopping theo loss thay vì AUC; một số trường hợp AUC còn tăng nhưng loss dao động.
	\item \textbf{Kết quả:} Val AUC $\approx 0.59$, Test AUC $\approx 0.63$.
	\item \textbf{Kết luận:} \textbf{7/10} -- cải thiện rõ rệt so với v1, còn có thể tinh chỉnh thêm theo AUC.
\end{itemize}

\subsubsection{\texttt{ViT-ResNet.ipynb} / ViT pretrained}

\begin{itemize}[noitemsep]
	\item \textbf{Mục tiêu:} Khai thác ViT-Base pretrained từ \texttt{timm} cho bài toán X-quang.
	\item \textbf{Kiến trúc:} \texttt{vit\_base\_patch16\_224}, patch size 16, 12 transformer layer, 12 head, embedding 768; thay head để có 15 đầu ra.
	\item \textbf{Kết quả:} Test AUC $\approx 0.6694$.
	\item \textbf{Vấn đề:} 86M tham số, tốn nhiều bộ nhớ, tốc độ huấn luyện chậm.
	\item \textbf{Kết luận:} \textbf{8/10} -- mô hình mạnh, rất tốt cho so sánh, nhưng nặng cho triển khai thực tế.
\end{itemize}

\subsubsection{\texttt{data.ipynb} và \texttt{config.py}}

\begin{itemize}[noitemsep]
	\item \textbf{\texttt{data.ipynb}:} Pipeline load dữ liệu hoạt động, nhưng phiên bản ban đầu chia theo ảnh (gây rò rỉ). Trong notebook Final đã chuyển sang chia theo bệnh nhân $\Rightarrow$ \textbf{tính đúng đắn đã được đảm bảo}.
	\item \textbf{\texttt{config.py}:} Gom cấu hình, nhưng còn hard-code đường dẫn Windows. Hướng cải thiện: dùng \texttt{pathlib.Path} và biến môi trường.
\end{itemize}

% ----------------------------------------------------------------------
\subsection{Bước 4: So sánh với bài báo gốc}

\subsubsection{Tóm tắt bài báo gốc}

Bài báo \cite{jain2024comparative}:
\begin{itemize}[noitemsep]
	\item So sánh CNN, ResNet và ViT cho phân loại đa lớp bệnh lý trên X-quang ngực.
	\item Dùng bộ NIH ChestX-ray14 (112{,}120 ảnh, 14 bệnh + No Finding).
	\item Báo cáo các độ chính xác (Accuracy) và AUC rất cao, ví dụ AUC khoảng 0.82--0.86.
\end{itemize}

\begin{table}[H]
	\centering
	\caption{Kết quả trong bài báo gốc (Jain et al., 2024)}
	\label{tab:original_paper_results}
	\begin{tabular}{lcccc}
		\toprule
		\textbf{Mô hình} & \textbf{Train Acc (\%)} & \textbf{Val Acc (\%)} & \textbf{Test Acc (\%)} & \textbf{AUC} \\
		\midrule
		CNN & 92.62 & 92.68 & 91.0 & 0.82 \\
		ResNet-34 & 93.38 & 93.34 & 93.0 & 0.86 \\
		ViT-v1/32 & 92.70 & 92.89 & 92.63 & 0.86 \\
		ViT-v2/32 & 92.94 & 92.95 & 92.83 & 0.84 \\
		ViT-ResNet/16 & 93.02 & 94.07 & \textbf{93.9} & 0.85 \\
		\bottomrule
	\end{tabular}
\end{table}

\subsubsection{Những điểm không được nêu rõ trong bài báo}

\begin{table}[H]
	\centering
	\caption{Một số khía cạnh bài báo gốc không ghi rõ và cách dự án xử lý}
	\label{tab:paper_vs_project}
	\begin{tabular}{lcc}
		\toprule
		\textbf{Khía cạnh} & \textbf{Bài báo gốc} & \textbf{Dự án này} \\
		\midrule
		Chiến lược chia dữ liệu & Không nêu rõ (suy ra chia theo ảnh) & \textbf{Chia theo bệnh nhân} \\
		Nguy cơ rò rỉ dữ liệu & Có (nhiều ảnh cùng bệnh nhân) & Được loại bỏ \\
		Framework & TensorFlow/Keras & PyTorch 2.x \\
		LR Scheduler & Không nói rõ & \texttt{ReduceLROnPlateau} \\
		Early stopping & Không đề cập & Có, theo dõi chỉ số validation \\
		Tính AUC & Không mô tả chi tiết & Xử lý lớp không đủ 0/1, macro AUC \\
		\bottomrule
	\end{tabular}
\end{table}

\subsubsection{Vì sao AUC của dự án thấp hơn bài báo gốc?}

Mô hình ViT từ đầu trong dự án đạt Test AUC = 0.7225, thấp hơn so với AUC 0.82--0.86 trong bài báo gốc. Lý do chính:

\begin{enumerate}[noitemsep]
	\item \textbf{Chia theo bệnh nhân:} Loại bỏ rò rỉ dữ liệu, do đó bài toán \textit{thực sự khó hơn}. Điểm AUC vì thế phản ánh đúng hơn khả năng tổng quát trên bệnh nhân mới.
	\item \textbf{Đánh giá nghiêm túc hơn:} Dự án sử dụng macro AUC, per-class AUC, và phân tích chi tiết từng lớp thay vì chỉ báo cáo Accuracy.
	\item \textbf{Khác biệt nhỏ về pipeline:} Framework khác (PyTorch vs TensorFlow), khác biệt trong mặc định optimizer, weight init, v.v.
\end{enumerate}

Từ góc nhìn khoa học, một mô hình đạt AUC $\approx 0.72$ trên tập bệnh nhân \textit{hoàn toàn mới} đáng tin cậy hơn nhiều so với AUC 0.85 nếu có khả năng rò rỉ dữ liệu giữa train/test.

% ----------------------------------------------------------------------
\subsection{Bước 5: Đánh giá bản thảo LaTeX}

File \texttt{Report/Group1\_Deeplearning.tex} hiện tại đã bao phủ đầy đủ các phần: Tóm tắt, Giới thiệu, Liên quan, Dataset, Phương pháp, Thực nghiệm, So sánh với bài báo, Kết luận và Phụ lục. Một số điểm cần cập nhật/nhấn mạnh thêm:

\begin{itemize}[noitemsep]
	\item Cập nhật bảng kết quả theo số liệu mới nhất (Test AUC = 0.7225 cho ViT từ đầu với chia theo bệnh nhân).
	\item Bổ sung bảng per-class AUC cho mô hình tốt nhất để minh họa khó khăn với các lớp hiếm.
	\item Nhấn mạnh rõ trong phần Phương pháp/Thực nghiệm rằng chia theo bệnh nhân được sử dụng để tránh rò rỉ dữ liệu.
\end{itemize}

Các nội dung này có thể được chèn vào các mục \emph{Experiments and Results}, \emph{Replication \& Differences} hoặc \emph{Appendix}.

% ----------------------------------------------------------------------
\subsection{Bước 6: Đánh giá tổng thể và khuyến nghị cuối cùng}

\subsubsection{Đánh giá theo tiêu chí ``research-grade''}

\begin{table}[H]
	\centering
	\caption{Đánh giá mức độ ``chuẩn nghiên cứu'' của dự án}
	\label{tab:research_grade}
	\begin{tabular}{lll}
		\toprule
		\textbf{Tiêu chí} & \textbf{Trạng thái} & \textbf{Ghi chú} \\
		\midrule
		Tái lập (reproducibility) & \checkmark & Seed cố định, config rõ ràng \\
		Thiết lập bài toán & \checkmark & Multi-label với sigmoid + BCE \\
		Xử lý dữ liệu & \checkmark & Chia theo bệnh nhân, tránh leakage \\
		Chỉ số đánh giá & \checkmark & Macro AUC, per-class AUC, không chỉ Accuracy \\
		So sánh baseline & \checkmark & CNN, ResNet, ViT-v1, ViT-v2, ViT pretrained \\
		Mô tả phương pháp & \checkmark & Có trong báo cáo LaTeX chi tiết \\
		Thảo luận hạn chế & \checkmark & Đã nêu trong phần Kết luận \\
		Chất lượng mã nguồn & \checkmark & Tổ chức rõ ràng, có chú thích \\
		\bottomrule
	\end{tabular}
\end{table}

\subsubsection{Kết luận của đợt rà soát}

Dựa trên toàn bộ phân tích trên, có thể kết luận rằng:

\begin{itemize}[noitemsep]
	\item Dự án đã đạt mức \textbf{sẵn sàng để nộp} cho một đồ án cao học với tiêu chuẩn nghiên cứu tốt.
	\item Các quyết định quan trọng về \textbf{thiết kế thực nghiệm} (đặc biệt là chia theo bệnh nhân) làm cho kết quả có ý nghĩa hơn về mặt lâm sàng, dù AUC có thấp hơn báo cáo gốc.
	\item Việc cung cấp đầy đủ mã nguồn, notebook, báo cáo LaTeX và mô tả chi tiết đảm bảo tính minh bạch và khả năng tái lập.
\end{itemize}

\textbf{Kết luận cuối cùng:} \emph{Dự án ViT-Chest-Xray ở trạng thái hiện tại đáp ứng tốt các tiêu chí nghiên cứu, có thể sử dụng làm đồ án cuối kỳ hoặc mở rộng thành bài báo khoa học với một số tinh chỉnh và thí nghiệm bổ sung.}
